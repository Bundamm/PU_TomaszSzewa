\documentclass[12pt, letterpaper, titlepage]{article}
\usepackage[left=3.5cm, right=2.5cm, top=2.5cm, bottom=2.5cm]{geometry}
\usepackage[MeX]{polski}
\usepackage[utf8]{inputenc}
\usepackage{graphicx}
\usepackage{enumerate}
\usepackage{amsmath} 
\usepackage{amssymb}
\usepackage{caption} 
\captionsetup[table]{labelformat=empty}
\title{Tabele}
\author{Tomasz Szewa}
\date{Październik 2022}
\begin{document}
\maketitle

\section{PIERWSZA TABELA}
\begin{table}[h]
\centering\caption{Tabela 1: Przykładowy system decyzyjny(U, A, d), modelujący problem diagnozy medycznej, której efektem jest decyzja o wykonaniu lub nie wykonaniu operacji wycięcia wyrostka robaczkowego,
U = \textit{{\{$u_1,u_2,...,u_{10}$}\}}, A = \textit{{\{$a_1,a_2$}\}}, d $\in$ D=\textit{{\{TAK,NIE}\}} \newline}
\begin{tabular}{c | c c c}
	\hline
	\hline
	Pacjent & Ból brzucha & Temperatura & Operacja\\
	\hline
	u1 & Mocny & Wysoka & Tak\\	
	
	u2 & Średni & Wysoka & Tak\\
	
	u3 & Mocny & Średnia & Tak\\
	
	u4 & Mocny & Niska & Tak\\
	
	u5 & Średni & Średnia & Tak\\
	
	u6 & Średni & Średnia & Nie\\
	
	u7 & Mały & Wysoka & Nie\\

	u8 & Mały & Niska & Tak\\
	
	u9 & Mocny & Niska & Tak\\
	
	u10 & Mały & Średnia & Tak\\
	\hline
	\hline

\end{tabular}
\end{table}
\section{BRAMKI LOGICZNE}
\begin{table}[h]
\centering\caption{NOT}
\begin{tabular}{| c | c |}
	\hline
	A & Q\\
	\hline
	0 & 1\\
	1 & 0\\
	\hline
\end{tabular}
\end{table}
\begin{table}[h]
\centering\caption{AND}
\begin{tabular}{| c c | c |}
	\hline
	A & B & Q\\
	\hline
	0 & 0 & 0\\
	0 & 1 & 0\\
	1 & 0 & 0\\
	1 & 1 & 1\\
	\hline
\end{tabular}
\end{table}
\begin{table}[h]
\centering\caption{NAND}
\begin{tabular}{| c c | c |}
	\hline
	A & B & Q\\
	\hline
	0 & 0 & 1\\
	0 & 1 & 1\\
	1 & 0 & 1\\
	1 & 1 & 0\\
	\hline
\end{tabular}
\end{table}
\begin{table}[h]
\centering\caption{OR}
\begin{tabular}{| c c | c |}
	\hline
	A & B & Q\\
	\hline
	0 & 0 & 0\\
	0 & 1 & 1\\
	1 & 0 & 1\\
	1 & 1 & 0\\
	\hline
\end{tabular}
\end{table}
\begin{table}[h]
\centering\caption{NOR}
\begin{tabular}{| c c | c |}
	\hline
	A & B & Q\\
	\hline
	0 & 0 & 1\\
	0 & 1 & 0\\
	1 & 0 & 0\\
	1 & 1 & 0\\
	\hline
\end{tabular}
\end{table}
\begin{table}[h]
\centering\caption{XOR}
\begin{tabular}{| c c | c |}
	\hline
	A & B & Q\\
	\hline
	0 & 0 & 0\\
	0 & 1 & 1\\
	1 & 0 & 1\\
	1 & 1 & 0\\
	\hline
\end{tabular}
\end{table}
\end{document}
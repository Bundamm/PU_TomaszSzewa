\documentclass[12pt, letterpaper,titlepage]{article}
\usepackage[left=3.5cm, right=2.5cm, top=2.5cm, bottom=2.5cm]{geometry}
\usepackage[MeX]{polski}
\usepackage[utf8]{inputenc}
\usepackage{graphicx}
\usepackage{enumerate}
\usepackage{amsmath} 
\usepackage{amssymb}
\title{Pierwszy dokument LaTeX}
\author{Tomek Szewa}
\date{Październik 2022}
\begin{document}
\maketitle
\section{Podstawowe Informacje}
\subsection{Imię i Nazwisko}
Tomasz Szewa
\subsection{Pochodzenie}
Mieszkam w Grudziądzu
\subsubsection{Dokładne Pochodzenie}
Urodziłem się natomiast w Świeciu.
\section{O mnie}
Lorem ipsum dolor sit amet, consectetur adipiscing elit. Morbi augue leo, ultrices ut nisl feugiat, scelerisque pellentesque purus. Vestibulum sagittis, diam id porta sagittis, diam erat ornare massa, a pulvinar arcu ligula sit amet est. Ut sagittis felis sit amet neque fermentum consectetur. Phasellus facilisis, purus ut ullamcorper ullamcorper, nisi neque tempus tellus, eget congue risus eros vel sem. Maecenas ornare, erat sit amet posuere gravida, neque sem varius odio, vitae tempus augue sapien quis sem. Vestibulum orci turpis, tristique ullamcorper sollicitudin eu, lobortis at arcu. Nullam id convallis nisi, id porttitor velit. Maecenas non mi non odio auctor ullamcorper at id felis. Suspendisse eu aliquet justo. Praesent ut odio faucibus, accumsan nibh quis, ullamcorper erat. Mauris erat elit, ultricies a ante vitae, facilisis tempus magna. Integer fermentum, nisi ut auctor auctor, arcu ligula pulvinar orci, ut venenatis risus massa non erat. Suspendisse non cursus eros. 
\begin{enumerate}[I.]
\item Aliquam urna libero.
\item Maecenas sagittis ligula.
\end{enumerate}
\subsection{Sekrety}
Aliquam urna libero, pellentesque sed feugiat id, aliquet eget nunc. Quisque dictum finibus consequat. Integer eleifend varius libero vel porta. Pellentesque vulputate ligula non urna elementum blandit. Vestibulum ut lobortis libero, congue mollis lacus. Maecenas fringilla, nisl ac posuere mollis, nisi risus facilisis orci, eget rutrum orci magna vitae sapien. Nam mattis velit vel metus placerat laoreet. Suspendisse porttitor varius dictum. Suspendisse a consectetur risus. Praesent a erat efficitur, posuere quam at, egestas quam. Aliquam consectetur id ipsum eget posuere. Nullam consequat velit quis metus rutrum commodo non quis urna. Cras sit amet libero sagittis, fermentum est non, sollicitudin neque. Maecenas turpis velit, euismod quis massa sit amet, interdum luctus quam. Proin sollicitudin purus at commodo venenatis. 
\newpage
\section{Przepis na szarlotkę.}
\subsection{SKŁADNIKI}
\textbf{Jabłka} 
\begin{enumerate}[•]
\item 1,5 kg jabłek (na szarlotkę najlepiej twardych i kwaśnych, np. szara reneta)
\item 5 łyżek cukru
\item 1/2 łyżeczki cynamonu
\end{enumerate}
\textbf{Ciasto}
\begin{enumerate}[•]
\item 300 g mąki
\item 250 g zimnego masła (50 g masła można zastąpić smalcem)
\item 1,5 łyżeczki proszku do pieczenia
\item 5 łyżek cukru
\item 1 łyżka cukru wanilinowego
\item 1 jajko
\item Do posypania: cukier puder
\end{enumerate}
\subsection{PRZYGOTOWANIE}
\subsubsection{JABŁKA}
\begin{enumerate}[•]
\item Jabłka obrać, pokroić na ćwiartki i wyciąć gniazda nasienne. Pokroić na mniejsze kawałki i włożyć do szerokiego garnka lub na głęboką patelnię.
\item Dodać cukier i cynamon i smażyć przez ok. 20 minut co chwilę mieszając, aż jabłka zmiękną i zaczną się rozpadać.
\end{enumerate}
\subsubsection{CIASTO}
\begin{enumerate}[•]
\item Do mąki dodać pokrojone w kostkę zimne masło, proszek do pieczenia, cukier i cukier wanilinowy. 
\item Składniki połączyć w jednolite ciasto (mikserem lub ręcznie), pod koniec dodać jajko (ciasto będzie dość miękkie).
\item Podzielić je na pół i włożyć obie połówki do zamrażarki na ok. 15 minut.
\end{enumerate}
\subsubsection{PIECZENIE}
\begin{enumerate}[•]
\item Piekarnik nagrzać do 180 st C. Przygotować niedużą formę*.
\item Wyjąć jedną połówkę ciasta z zamrażarki, pokroić nożem na plasterki i wylepić nimi spód formy. Następnie wyłożyć na to jabłka.
\item Pozostałe ciasto zetrzeć na tarce bezpośrednio na jabłka (lub pokroić ciasto na plasterki i ułożyć na wierzchu).
\item Wstawić do piekarnika i piec przez ok. 50 minut lub na złoty kolor. Upieczoną szarlotkę przestudzić i posypać cukrem pudrem.
\end{enumerate}
\subsection{WSKAZÓWKI}
 * np. prostokątną formę 21 x 27 cm lub kwadratową 24 x 24 cm lub tortownicę o średnicy 26 cm 
\end{document}